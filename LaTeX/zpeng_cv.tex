%!TEX TS-program = xelatex
%!TEX encoding = UTF-8 Unicode

%-------------------------------------------------------------------------------
% CONFIGURATIONS
%-------------------------------------------------------------------------------
% A4 paper size by default, use 'letterpaper' for US letter
\documentclass[12pt, letterpaper]{awesome-cv}

\RequirePackage[backend=biber,style=ieee,sorting=none]{biblatex}

%% For removing numbering entirely when using a numeric style
%\setlength{\bibhang}{1em}
%\DeclareFieldFormat{labelnumberwidth}{\makebox[\bibhang][l]{\itemmarker}}
% \setlength{\biblabelsep}{0pt}
%\defbibheading{pubtype}{\cvsubsection{#1}}
%\renewcommand{\bibsetup}{\vspace*{-\baselineskip}}
%\AtEveryBibitem{\makebox[\bibhang][l]{\itemmarker}}
%\setlength{\bibitemsep}{0.25\baselineskip}

% For fundamental structures
\renewcommand*{\headerfirstnamestyle}[1]{{\fontsize{24pt}{1em}\headerfont\color{graytext} #1}}
\renewcommand*{\headerlastnamestyle}[1]{{\fontsize{24pt}{1em}\headerfont\bfseries\color{text} #1}}
\renewcommand*{\headerpositionstyle}[1]{{\fontsize{10pt}{1em}\bodyfont\scshape\color{awesome} #1}}
\renewcommand*{\headeraddressstyle}[1]{{\fontsize{9pt}{1em}\headerfont\itshape\color{lighttext} #1}}
\renewcommand*{\headersocialstyle}[1]{{\fontsize{9pt}{1em}\headerfont\color{text} #1}}

\renewcommand*{\paragraphstyle}{\fontsize{11pt}{1.2em}\bodyfont\upshape\color{text}}

% For elements of entry
\renewcommand*{\entrytitlestyle}[1]{{\fontsize{12pt}{1em}\bodyfont\bfseries\color{darktext} #1}}
\renewcommand*{\entrypositionstyle}[1]{{\fontsize{12pt}{1em}\bodyfont\scshape\color{graytext} #1}}
\renewcommand*{\entrydatestyle}[1]{{\fontsize{11pt}{1em}\bodyfont\slshape\color{graytext} #1}}
\renewcommand*{\entrylocationstyle}[1]{{\fontsize{11pt}{1em}\bodyfont\slshape\color{awesome} #1}}
\renewcommand*{\descriptionstyle}[1]{{\fontsize{11pt}{1.2em}\bodyfont\upshape\color{text} #1}}

% For elements of subentry
\renewcommand*{\subentrytitlestyle}[1]{{\fontsize{12pt}{1em}\bodyfont\mdseries\color{graytext} #1}}
\renewcommand*{\subentrypositionstyle}[1]{{\fontsize{12pt}{1em}\bodyfont\scshape\color{graytext} #1}}
\renewcommand*{\subentrydatestyle}[1]{{\fontsize{11pt}{1em}\bodyfont\slshape\color{graytext} #1}}
\renewcommand*{\subentrylocationstyle}[1]{{\fontsize{11pt}{1em}\bodyfont\slshape\color{awesome} #1}}
\renewcommand*{\subdescriptionstyle}[1]{{\fontsize{11pt}{1em}\bodyfont\upshape\color{text} #1}}

% For elements of honor
\renewcommand*{\honortitlestyle}[1]{{\fontsize{11pt}{1em}\bodyfont\color{graytext} #1}}
\renewcommand*{\honorpositionstyle}[1]{{\fontsize{11pt}{1em}\bodyfont\bfseries\color{darktext} #1}}
\renewcommand*{\honordatestyle}[1]{{\fontsize{11pt}{1em}\bodyfont\color{graytext} #1}}
\renewcommand*{\honorlocationstyle}[1]{{\fontsize{11pt}{1em}\bodyfont\slshape\color{awesome} #1}}

% For elements of skill
\renewcommand*{\skilltypestyle}[1]{{\fontsize{11pt}{1em}\bodyfont\bfseries\color{darktext} #1}}
\renewcommand*{\skillsetstyle}[1]{{\fontsize{11pt}{1em}\bodyfont\color{text} #1}}

% Configure page margins with geometry
\geometry{left=2.5cm, top=2cm, right=2.5cm, bottom=2cm, footskip=.5cm}

% Specify the location of the included fonts
\fontdir[fonts/]

% Color for highlights
% Awesome Colors: awesome-emerald, awesome-skyblue, awesome-red, awesome-pink, awesome-orange
%                 awesome-nephritis, awesome-concrete, awesome-darknight
% \colorlet{awesome}{awesome-red}
% Uncomment if you would like to specify your own color
\definecolor{awesome}{HTML}{1976D2}

% Colors for text
% Uncomment if you would like to specify your own color
%\definecolor{darktext}{HTML}{414141}
\definecolor{text}{HTML}{333333}
\definecolor{graytext}{HTML}{333333}
\definecolor{lighttext}{HTML}{333333}

% Set false if you don't want to highlight section with awesome color
\setbool{acvSectionColorHighlight}{true}

% If you would like to change the social information separator from a pipe (|) to something else
\renewcommand{\acvHeaderSocialSep}{\quad\textbar\quad}


%-------------------------------------------------------------------------------
%	PERSONAL INFORMATION
%	Comment any of the lines below if they are not required
%-------------------------------------------------------------------------------
% Available options: circle|rectangle,edge/noedge,left/right
% \photo{./examples/profile.png}
\name{Zhengyu}{Peng}
\position{Senior Radar Systems Engineer{\enskip\cdotp\enskip}Ph.D.}
\address{925 E Edenbridge Way, Westfield, IN 46074, USA}

\mobile{(+1) 806-392-6110}
\email{zpeng.me@gmail.com}
\homepage{zpeng.me}
\github{rookiepeng}
\googlescholar{0vQPboMAAAAJ}{Z. Peng}
% \linkedin{posquit0}
% \gitlab{gitlab-id}
% \stackoverflow{SO-id}{SO-name}
% \twitter{@twit}
% \skype{skype-id}
% \reddit{reddit-id}
% \extrainfo{extra informations}

%\quote{``Be the change that you want to see in the world."}

% bibliography with mutiple entries
%\usepackage[resetlabels]{multibib}
%\newcites{book}{Books}
%\newcites{journal}{Journal Articles}
%\newcites{proceeding}{Conference Proceedings}
%\newcites{patent}{Patents}

\addbibresource{cv/books.bib}
\addbibresource{cv/journals.bib}
\addbibresource{cv/proceedings.bib}
\addbibresource{cv/patents.bib}

%-------------------------------------------------------------------------------
\begin{document}

% Print the header with above personal informations
% Give optional argument to change alignment(C: center, L: left, R: right)
\makecvheader

% Print the footer with 3 arguments(<left>, <center>, <right>)
% Leave any of these blank if they are not needed
\makecvfooter
    {\today}
    {Zhengyu~Peng}
    {\thepage}

%-------------------------------------------------------------------------------
%	BIBLIOGRAPHY
%-------------------------------------------------------------------------------

%-------------------------------------------------------------------------------
%	CV/RESUME CONTENT
%	Each section is imported separately, open each file in turn to modify content
%-------------------------------------------------------------------------------
\cvsection{Summary}

\cvparagraph{
    Dr. Zhengyu Peng is a highly experienced radar systems engineer with expertise in automotive radar, signal processing, and antenna arrays. He earned his Ph.D. from Texas Tech University and has contributed to the development of high-resolution imaging radar technology at Aptiv, a global technology company. Dr. Peng is also an associate editor for the IEEE Transactions on Instrumentation and Measurement.
}


%-------------------------------------------------------------------------------
%	SECTION TITLE
%-------------------------------------------------------------------------------
\cvsection{Experience}

%-------------------------------------------------------------------------------
%	CONTENT
%-------------------------------------------------------------------------------
\begin{cventries}

%---------------------------------------------------------
  \cventry
    {Senior Radar Systems Engineer} % Job title
    {Aptiv} % Organization
    {Kokomo, IN} % Location
    {Jun. 2018 - PRESENT} % Date(s)
    {
      \begin{cvitems} % Description(s) of tasks/responsibilities
        \item {Design and build next generation radar systems for active safety and autonomous vehicles}
      \end{cvitems}
    }

%---------------------------------------------------------
  \cventry
    {Research Internship} % Job title
    {Mitsubishi Electric Research Labs. (MERL)} % Organization
    {Cambridge, MA} % Location
    {May 2017 - Aug. 2017} % Date(s)
    {
      \begin{cvitems} % Description(s) of tasks/responsibilities
        \item {Designed novel digital beamforming transmitter architectures for radars and communication systems aiming to reduce hardware complexity and power consumption}
        \item {Completed simulation evaluation and initial schematic design}
        %\item {Outcomes:
        %\begin{itemize}%
        %    \item 1 conference paper presented in EuCAP 2018
        %    \item 2 patents filed
        %\end{itemize}
        %}
      \end{cvitems}
    }

%---------------------------------------------------------
\end{cventries}

%-------------------------------------------------------------------------------
%	SECTION TITLE
%-------------------------------------------------------------------------------
\cvsection{Education}


%-------------------------------------------------------------------------------
%	CONTENT
%-------------------------------------------------------------------------------
\begin{cventries}

%---------------------------------------------------------
  \cventry
    {Ph.D. in Electrical Engineering} % Degree
    {Texas Tech University} % Institution
    {Lubbock, TX} % Location
    {Aug. 2014 - May 2018} % Date(s)
    {
      %\begin{cvitems} % Description(s) bullet points
      %  \item {Got a Chun Shin-Il Scholarship which is given to promising students in CSE Dept.}
      %\end{cvitems}
    }\vspace{-1.2em}
    
%---------------------------------------------------------
  \cventry
    {M.Sc. in Information
Science and Electronic Engineering} % Degree
    {Zhejiang University} % Institution
    {Hangzhou, China} % Location
    {Aug. 2011 - Mar. 2014} % Date(s)
    {
      %\begin{cvitems} % Description(s) bullet points
      %  \item {Got a Chun Shin-Il Scholarship which is given to promising students in CSE Dept.}
      %\end{cvitems}
    }\vspace{-1.2em}
    
%---------------------------------------------------------
  \cventry
    {B.Sc. in Information
Science and Electronic Engineering} % Degree
    {Zhejiang University} % Institution
    {Hangzhou, China} % Location
    {Aug. 2007 - Jun. 2011} % Date(s)
    {
      %\begin{cvitems} % Description(s) bullet points
      %  \item {Got a Chun Shin-Il Scholarship which is given to promising students in CSE Dept.}
      %\end{cvitems}
    }\vspace{-1.2em}

%---------------------------------------------------------
\end{cventries}

\cvsection{Skills}

\begin{cvskills}

    \cvskill
        {Actively Using}
        {Python, MATLAB, C++, CUDA, Git, CANape}

    \cvskill
        {Experience With}
        {Java, C, Verilog, CST Microwave Studio, Keysight ADS, Cadence Virtuoso/Allegro}

\end{cvskills}

\newpage
\cvsection{Projects}

Featured research projects at Aptiv

\begin{cvskills}
    \cvskill
        {FLR7HD}
        {Next-generation high-resolution 4D imaging radar.}

    \cvskill
        {FLR4+}
        {Aptiv’s first 4D imaging radar in production for active safety and autonomous driving.}

    \cvskill
        {ISR}
        {Interior sensing radar for detecting small child or baby being left inside a vehicle.}

\end{cvskills}

Featured research projects at Texas Tech University (details on \href{https://zpeng.me}{https://zpeng.me})

\begin{cvskills}
    \cvskill
        {3D MIMO radar}
        {A portable 24-GHz 3D MIMO radar system.}

    \cvskill
        {Phased array radar}
        {A short-range localization radar with beamforming capability in K-band.}

    \cvskill
        {Multi-Mode Radar}
        {A portable K-band radar for short-range localization and vital sign detection.}

\end{cvskills}

Featured personal projects (details on \href{https://zpeng.me}{https://zpeng.me})

\begin{cvskills}

    \cvskill
        {RadarSimPy}
        {A radar simulator built with Python and C++.}

    \cvskill
        {SensorView}
        {A lightweight sensor data visualization and analysis tool.}

    \cvskill
        {Antenna array analysis}
        {A simple GUI tool for antenna array analysis.}

    % \cvskill
    %     {Antenna models}
    %     {Antenna simulation models in my researches and projects.}

    % \cvskill
    %     {Microwave structures}
    %     {Design files of microwave structures in my researches and projects.}

\end{cvskills}

\cvsection{Honors \& Awards}

%\cvsubsection{International}

\begin{cvhonors}

  \cvhonor
    {Outstanding Reviewer}
    {IEEE Instrumentation and Measurement Society}
    {--}
    {2018}

  \cvhonor
    {Travel Fellowship}
    {U.S. National Committee for the International Union of Radio Science}
    {Boulder, CO}
    {2018}

  \cvhonor
    {Horn Professor’s Graduate Achievement Award}
    {Texas Tech University}
    {Lubbock, TX}
    {2017}

  \cvhonor
    {Graduate Fellowship}
    {IEEE Microwave Theory and Techniques Society}
    {San Francisco, CA}
    {2016}

  \cvhonor
    {Finalist}
    {IEEE Radio Wireless Week Student Paper Competition}
    {Austin, TX}
    {2016}

  \cvhonor
    {Excellent Demo Track}
    {IEEE Radio Wireless Week}
    {Austin, TX}
    {2016}

  \cvhonor
    {Third Place}
    {IEEE IMS High Sensitivity Radar Competition}
    {Phoenix, AZ}
    {2015}

\end{cvhonors}


\cvsection{Professional Activities}

\cvsubsection{Journal Reviewer}

\cvparagraph{
    \begin{cvitems}
        \vspace{4.0mm}
        \item {Scientific Reports }
        \item {IEEE Sensors Letters }
        \item {IEEE/ASME Transactions on Mechatronics }
        \item {IEEE Transactions on Biomedical Engineering }
        \item {International Journal of Microwave and Wireless Technologies }
        \item {IEEE Access }
        \item {IEEE Transactions on Microwave Theory and Techniques }
        \item {IEEE Transactions on Instrumentation and Measurement }
        \item {IEEE Transactions on Circuits and Systems I: Regular Papers }
        \item {IEEE Transactions on Circuits and Systems II: Express Briefs }
        \item {IEEE Transactions on Mobile Computing }
        \item {IEEE Transactions on Vehicular Technology }
        \item {IEEE Antennas and Wireless Propagation Letters }
        \item {IEEE Microwave and Wireless Components Letters }
        \item {IEEE Microwave Magazine }
        \item {IEEE Sensors Journal }
        \item {IEEE Journal of Electromagnetics, RF, and Microwaves in Medicine and Biology }
        \item {IEEE Journal on Emerging and Selected Topics in Circuits and Systems }
        \item {IETE Journal of Research }
        \item {Sensors and Actuators A: Physical }
        \item {Sensors }
        \item {Electronics }
        \item {Remote Sensing }
        \item {Algorithms }
        \item {Applied Sciences }
        \item {Symmetry }
        \item {Information }
        \item {Mathematical and Computational Applications }
        \item {Advances in Science, Technology and Engineering Systems Journal }
        \item {Computers in Biology and Medicine }
        \item {Engineering Applications of Artificial Intelligence }
        \item {Expert Systems With Applications }
        \item {AEÜ - International Journal of Electronics and Communications }
        \item {Wind Energy }
        \item {ACES Journal }
        
        \vspace{4.0mm}
    \end{cvitems}
}

\cvsubsection{Conference Technical Program Committee Reviewer}

\cvparagraph{
    \begin{cvitems}
        \vspace{4.0mm}
        \item {2019 International Applied Computational Electromagnetics Society (ACES) Symposium }
        \item {2018 IEEE International RF and Microwave Conference }
        \item {2018 World of Multidisciplinary Research and Application Conference }
        \item {2018 Advanced Research in Eng. and Info. Technology International Conference }
        \item {2018 Symposium on Islamic Sciences and Technology }
        \item {2018 World Congress on Circuits and Systems Conference }
        \item {2017 Asia Pacific Microwave Conference }
        
        \vspace{4.0mm}
    \end{cvitems}
}

\cvsection{Publications}

Citations: 1241, h-index: 20, i10-index: 29 (Recorded on Nov. 05, 2022)

\cvsubsection{Book}

\begin{refsection}
	\nocite{Radar2022}
	{\color{text}\printbibliography[heading=none, type=book]}
\end{refsection}

\cvsubsection{Book Chapters}

\begin{refsection}
	\nocite{Peng2020, Peng2019, Gomez-Garcia2018}
	{\color{text}\printbibliography[heading=none, type=book]}
\end{refsection}

\cvsubsection{Journal Articles}
\begin{refsection}
	\nocite{TIM2018, Sensors2019, SJ2018, TMTT2018SAR, TMTT2018MIMO, TMTT2017Array, JERM2017review, TMTT2017review, TMTT2017FMCW, TIM2016, Sensors2016, TMTT2016, TMTT2016Beamforming, MM2015, TMTT2014, AWPL2013}
	{\color{text}\printbibliography[heading=none, type=article]}
\end{refsection}

\cvsubsection{Conference Proceedings}
\begin{refsection}
	\nocite{NDSS2023, TELSIKS2019, IMBioC2019, WiSNet2019, DCAS2018, Sensys2018, ACES2018, APS2018, RASC2018, I2MTC20182, I2MTC20181, IWS2018, EuCAP2018, NRSM2018, APMC2017, ACES2017, ACES20172, ACES20171, IMBIOC2017, WiSNet20171, WiSNet20172, NRSM2017, RFIT2016, IMS20161, IMS2016, IWS2016, IWS20161, BioWireleSS2016, BioWireleSS20161, RWS2016, IMWS2015, MWSCAS2015}
	{\color{text}\printbibliography[heading=none, type=inproceedings]}
\end{refsection}

\newpage

\cvsubsection{Patents}
\begin{refsection}
	\nocite{SLOWTIME2022, FREQUENCY2022, OBJECT2021, COMPLEX2021, DIGITAL2019, ANALOG2019, OMNI2015, ARRAY2015, TRAY2013}
	{\color{text}\printbibliography[heading=none, type=patent]}
\end{refsection}


%-------------------------------------------------------------------------------
\end{document}
